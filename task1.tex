\documentclass[12pt]{article}

\usepackage[russian]{babel}
\usepackage{amsmath}

\title{Домашняя работа №1}
\author{Решетков Андрей}
\date{}

\begin{document}
	\maketitle
	\begin{flushright}
		\textit{
			Audi multa,\\
			loquere pauca
		}
	\end{flushright}
	
	\vspace{20pt}
	Это мой первый документ в системе компьютерной вёрстки \LaTeX.
	
	\begin{center}
		\huge \sffamily <<Ура!!!>>	
	\end{center}

	А теперь формулы. \textsc{Формула}~--- краткое и точное словесное выражение, определение или же ряд математических величин, выраженный условными знаками.
	
	\vspace{15pt}
	\hspace{28pt}{\Large\textbf{Термодинамика}}
	
	Уравнение Менделеева--Клапейрона~--- уравнение сосояния идеального газа, имеющее вид $pV = \nu RT$, где $p$~--- давление, $V$~--- объём, занимаемый газом, $T$~--- температура газа, $\nu$~--- количество вещества газа, а $R$~--- универсальная газовая постоянная.
	
	\vspace{15pt}
	\hspace{28pt}{\Large\textbf{Геометрия \hfill Планиметрия}}
	%\textbf{
	%	\hspace{15pt} \Large {
	%	Геометрия \hfill Планиметрия}
	%}
	
	Для плоского треугольника со сторонами $a, b, c$ и углом $\alpha$, лежащим против стороны $a$, справедливо соотношение
	$$
		a^2 = b^2 + c^2 - 2bc\cos\alpha,
	$$
	из которого можно выразить косинус угла треугольника:
	$$
		\cos\alpha = \frac{b^2 + c^2 - a^2}{2bc}.
	$$
	
	Пусть $p$~--- полупериметр треугольника, тогда путем несложных преобразований можно получить, что
	$$
		\tg\frac{\alpha}{2} = \sqrt{\frac{(p-b)(p-c)}{p(p-a)}}.
	$$	
	\\[1cm]
	На сегодня, пожалуй, хватит\dots Удачи!

\end{document}